% Copyright (C)  2015  Alexander Jankowski, Philipp Hacker.
% Permission is granted to copy, distribute and/or modify this document
% under the terms of the GNU Free Documentation License, Version 1.3
% or any later version published by the Free Software Foundation;
% with no Invariant Sections, no Front-Cover Texts, and no Back-Cover Texts.
% The lincense itself can be found at <https://www.gnu.org/licenses/fdl-1.3>.

\documentclass[numbers=noenddot,a4paper,notitlepage,twoside,BCOR15mm]{scrartcl}
%\documentclass[numbers=noenddot,12pt,a4paper]{scrartcl}

\usepackage{ifoddpage}
\usepackage[infoshow]{tabularx}
\usepackage{fancyhdr}
\usepackage[greek,ngerman]{babel}
\usepackage[T1]{fontenc}
\usepackage[utf8]{inputenc}
\usepackage{libertine}
\usepackage{ziffer}
\usepackage{graphicx}
\usepackage{units}
\usepackage[infoshow]{tabularx}
\usepackage[all]{xy}
\usepackage{amsmath}
\usepackage{amssymb}
\usepackage{wrapfig}
\usepackage{upgreek}
\usepackage{esint}
\usepackage{float}
\usepackage[font=small,labelfont=bf]{caption}
\usepackage{subcaption}
\usepackage{lscape}
\usepackage[backref=page]{hyperref}
\usepackage{cleveref}
\usepackage{csquotes}

\renewcommand{\headrulewidth}{0.1pt}
\renewcommand{\footrulewidth}{0.1pt}
\newcommand{\name}{\text{Name}} %TODO Name des Protokollanten eintragen

\renewcaptionname{ngerman}{\figurename}{Abb. }
\renewcaptionname{ngerman}{\tablename}{Tab.}

\setlength{\parindent}{0pt}

\newcommand{\nummat}[1]{\left[\text{#1}\right]}
\newcommand{\num}[1]{$\left[\text{#1}\right]$}
\newcommand{\degree}{^\circ}
\newcommand{\diff}{\textnormal{d}}
\newcommand{\tenpo}[1]{ 10^{#1}}
\newcommand{\greek}[1]{\greektext#1\latintext}
\newcommand{\ix}[1]{_\text{#1}}
\newcommand{\imag}{\mathbf{i}}
\newcommand{\tilt}[1]{\textit{#1}}
\newcommand{\grad}[1]{\textit{grad}\left(#1\right)}
\newcommand{\divergenz}[1]{\textit{div}\left(#1\right)}
\newcommand{\euler}{\mathnormal{e}}
\newcommand{\fett}[1]{\textbf{#1}}
\newcommand{\einnup}{\hspace{0.2cm}}
\newcommand{\einnum}{\hspace{-0.2cm}}
\newcommand{\zentriert}[1]{\begin{center}#1\end{center}}

\title{Protokoll: Positronen und Koinzidenz-Messungen} %TODO Name des Versuchs eintragen
\author{Alexander Jankowski, Philipp Hacker}
\date{\today}
\pagestyle{fancy}
\fancyhead[C]{\thepage}
\fancyhead[R]{\name}
\fancyfoot[C]{\thepage}
\fancyhead[L]{Abschnitt \thesection}

\begin{document}
	\maketitle
	\begin{center}
		Betreuer: Gerrit Marx\\ %TODO Name des Betreuers eintragen
		Versuchsdatum: 21./22./23.10.2015\\ %TODO Datum des Versuchs eintragen
		\begin{table}[h]
			\centering
			Note: %TODO Gute Note erhalten :)
			\begin{tabularx}{1.5cm}{|X|}
				\hline \\ \\
				\hline
			\end{tabularx}
		\end{table}
	\end{center}
	\vspace*{\fill}
	\tableofcontents
	\vfill
	\newpage
	\section{Motivation}
	
	\newpage
	\section{Physikalische Grundlagen}
	\subsection{Erzeugung und Vernichtung von Positronen}

		Das Positron ist ein stabiles Lepton mit der Leptonenzahl $L=-1$, dem Spin $s=\frac{1}{2}$ und das Anti-Teilchen des Elektrons. Somit gehört es, dem Standardmodell zur Folge, zu den Elementarteilchen und damit den grundlegenden Bausteinen der Materie. Es wird durch das Symbol $\textbf{e}^+$ ausgedrückt. Dessen Ladung ist die entgegensetzte zum Elektron $q=1\cdot\text{e}=\unit[1,602 176462(63)\cdot\tenpo{-19}]{C}$ (ebenso magnetisches Moment), seine Masse ist gleich der des $\textbf{e}^-$ $m_{\textbf{e}^+}=\unit[9,10938188(72)\cdot\tenpo{-31}]{kg}$ und auch in allen anderen Eigenschaften stimmen die beiden Teilchen überein. Es wurde 1928 vom Nobelpreisträger \tilt{Paul M. Dirac} vorhergesagt und erstmal 1932 von \tilt{Carl David Anderson} in der kosmischen Strahlung nachgewiesen. Das Positron tritt also u.a. beim Zerfall positiver Myonen in der Atmosphäre, dem $\beta^+$-Zerfall mit der Form
		
		\begin{align}
			^{X}_{Z}A \rightarrow ^{X}_{Z-1}B+\textbf{e}^{+}+\nu\ix{e}
		\end{align}

		oder bei hochenergetischen Stoßprozessen bzw. der Proton-Proton-Wechselwirkung.\\
		Treffen Teilchen und Anti-Teilchen aufeinander ($\textbf{e}^+$ und $\textbf{e}^-$), so kommt es zu deren Wechselwirkung in Form der Bildung des \tilt{Positroniums} - sozusagen dem "`Anti-Atom"' zum Wasserstoff, welches jedoch sehr kurzlebig ist. Sind die Streupartner zu schnell oder langsam (der Wirkungsquerschnitt ist stark energieabhängig), kommt es, ohne Ausbildung und Zerfall eines gebundenen Zustandes vorher, zur Paarvernichtung. Dabei \tilt{annihilieren} sich Elektron und Positron unter Emission von 2 Photonen mit Gesamtenergie der kinetischen Energie beider Teilchen und deren Ruhemassen. Unter Voraussetzung, dass das Elektron stationär und das Positron stark verlangsamt ist, ergibt sich eine Photonen-Energie von $\unit[511]{keV}$. Der Winkel, unter welchen diese sich auseinander bewegen, wird, mit den selben Annahmen, zu $\sim\unit[180]{\degree}$.

	\newpage
	\section{Durchführung}
	
	\newpage
	\section{Auswertung}
	
	\newpage
	\section{Anhang}

		\bibliography{all.bib}
		\bibliographystyle{unsrt}

\end{document}