% Copyright (C)  2015  Alexander Jankowski, Philipp Hacker.
% Permission is granted to copy, distribute and/or modify this document
% under the terms of the GNU Free Documentation License, Version 1.3
% or any later version published by the Free Software Foundation;
% with no Invariant Sections, no Front-Cover Texts, and no Back-Cover Texts.
% The lincense itself can be found at <https://www.gnu.org/licenses/fdl-1.3>.

\documentclass[numbers=noenddot,a4paper,notitlepage,twoside,BCOR15mm]{scrartcl}
%\documentclass[numbers=noenddot,12pt,a4paper]{scrartcl}

\usepackage{ifoddpage}
\usepackage[infoshow]{tabularx}
\usepackage{fancyhdr}
\usepackage[greek,ngerman]{babel}
\usepackage[T1]{fontenc}
\usepackage[utf8]{inputenc}
\usepackage{libertine}
\usepackage{ziffer}
\usepackage{graphicx}
\usepackage{units}
\usepackage[infoshow]{tabularx}
\usepackage[all]{xy}
\usepackage{amsmath}
\usepackage{amssymb}
\usepackage{wrapfig}
\usepackage{upgreek}
\usepackage{esint}
\usepackage{float}
\usepackage[font=small,labelfont=bf]{caption}
\usepackage{subcaption}
\usepackage{lscape}
\usepackage[backref=page]{hyperref}
\usepackage{cleveref}
\usepackage{csquotes}

\renewcommand{\headrulewidth}{0.1pt}
\renewcommand{\footrulewidth}{0.1pt}
\newcommand{\name}{\text{Philipp Hacker}} %TODO Name des Protokollanten eintragen

\renewcaptionname{ngerman}{\figurename}{Abb. }
\renewcaptionname{ngerman}{\tablename}{Tab.}

\setlength{\parindent}{0pt}

\newcommand{\nummat}[1]{\left[\text{#1}\right]}
\newcommand{\num}[1]{$\left[\text{#1}\right]$}
\newcommand{\degree}{^\circ}
\newcommand{\diff}{\textnormal{d}}
\newcommand{\tenpo}[1]{ 10^{#1}}
\newcommand{\greek}[1]{\greektext#1\latintext}
\newcommand{\ix}[1]{_\text{#1}}
\newcommand{\imag}{\mathbf{i}}
\newcommand{\tilt}[1]{\textit{#1}}
\newcommand{\grad}[1]{\textit{grad}\left(#1\right)}
\newcommand{\divergenz}[1]{\textit{div}\left(#1\right)}
\newcommand{\euler}{\mathnormal{e}}
\newcommand{\fett}[1]{\textbf{#1}}
\newcommand{\einnup}{\hspace{0.2cm}}
\newcommand{\einnum}{\hspace{-0.2cm}}
\newcommand{\zentriert}[1]{\begin{center}#1\end{center}}

\title{Protokoll: Mie-Streuung} %TODO Name des Versuchs eintragen
\author{Alexander Jankowski, Philipp Hacker}
\date{\today}
\pagestyle{fancy}
\fancyhead[C]{\thepage}
\fancyhead[R]{\name}
\fancyfoot[C]{\thepage}
\fancyhead[L]{Abschnitt \thesection}

\begin{document}
	\maketitle
	\begin{center}
		Betreuer: Malte Paßvogel\\ %TODO Name des Betreuers eintragen
		Versuchsdatum: 28./29.10.2015\\ %TODO Datum des Versuchs eintragen
		\begin{table}[h]
			\centering
			Note: %TODO Gute Note erhalten :)
			\begin{tabularx}{1.5cm}{|X|}
				\hline \\ \\
				\hline
			\end{tabularx}
		\end{table}
	\end{center}
	\vspace*{\fill}
	\tableofcontents
	\vfill
	\newpage
	\section{Motivation}

	\newpage
	\section{Physikalische Grundlagen}

		\subsection{Streutheorie elektromagnetischer Wellen nach Mie}\label{subsec:miestreu}

			Die Theorie der \tilt{Mie-Streuung} (oder auch \tilt{Lorenz-Mie-Streuung}) wurde von Gustav Mie 1908 - während seiner Professur an der Ernst-Moritz-Arndt-Universität in Greifswald - eingeführt und beschäftigt sich mit der elastischen Streuung von elektromagnetischen Wellen an sphärischen Objekten, deren Dimension i.A. etwa der Länge eines Wellenzuges entspricht. Konkreter handelt es sich dabei um die exakte Lösung der n Maxwell-Gleichungen für ebene Lichtwellen, die auf ein Teilchen treffen, welches unter der Einwirkung dieser externen Felder Mutlipolmomente der Elektronenkonfiguration ausbildet. Man versteht - grob gesagt - die einfallende elektromagnetische Welle als Störung der ruhenden Elektronenwolke des \tilt{targets}, welche dann, unter Reflexion, Absorption und Brechung wieder eine Welle emittiert, die mit den Multipolmomenten moduliert ist. Schließlich erhält man als ausfallende Welle abstrahlende, sphärische Wellenfunktionen (e.g. Kugelflächenfunktionen) und als Feld des Partikels einfache sphärische Flächenfunktionen. Die damit an jedem Raumpunkt berechenbare Welle ergibt sich als Interferenzerscheinung der vom Objekt gebeugten Teilwellen, wie sie die Ladungsträgerschwankungen moduliert hatten.\\
			Die Mie-Streuung ist, in Relationen zur Wellenlänge $\lambda$ des eingestrahlten Lichtes, im Bereich von $\unit[0,2]{\lambda}<R<\unit[2]{\lambda}$ des Radius $R$ des Partikels eine gute Möglichkeit, das Lichtfeld um z.B. ein Kolloid zu beschreiben. Dementsprechend kann man aus der Untersuchung dessen auf verschiedenste Eigenschaften des Objektes schließen: Elektronendichten bzw. -verteilungen, Größe, Beschaffenheit und daraus folgend Brechungsindex, Absorption usw.\\
			Um der Theorie eine mathematische Struktur zu geben, seien einige Ausdrücke für spektroskopische Größen angeführt.\\
			Die Absorption $\varepsilon\ix{Im}=\text{Im}\left(\varepsilon\left(\omega\right)\right)$ und die Polarisation einer Welle $\varepsilon\ix{Re}=\text{Re}\left(\varepsilon\left(\omega\right)\right)$ sind die Anteile der dielektrischen Funktion $\varepsilon\left(\omega\right)$ des Partikels, wobei $\omega$ die Frequenz des Eingestrahlten Lichtes ist, $\omega\ix{P}$ die Plasma-Frequenz und $\Gamma=v\ix{F}/\lambda\ix{mfp}$ die Dämpfung mit der Fermi-Geschwindigkeit $v\ix{F}$.
			
				\begin{align}
					\varepsilon\left(\omega\right)=1-\frac{\omega\ix{P}^2}{\omega^2-\imag\Gamma\omega}
				\end{align}

		\subsection{Atomic-Force-Microscopy (Rasterkraftmikroskop)}\label{subsec:afm}

		\subsection{Röntgenreflektion}\label{subsec:röntgen}


	\newpage
	\section{Durchführung}

	\newpage
	\section{Auswertung}

	\newpage
	\section{Anhang}

		\bibliography{all.bib}
		\bibliographystyle{unsrt}

\end{document}