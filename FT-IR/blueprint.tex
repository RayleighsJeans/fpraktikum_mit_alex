% Copyright (C)  2015  Alexander Jankowski, Philipp Hacker.
% Permission is granted to copy, distribute and/or modify this document
% under the terms of the GNU Free Documentation License, Version 1.3
% or any later version published by the Free Software Foundation;
% with no Invariant Sections, no Front-Cover Texts, and no Back-Cover Texts.
% The lincense itself can be found at <https://www.gnu.org/licenses/fdl-1.3>.

\documentclass[numbers=noenddot,a4paper,10pt,twocolumn]{article}
%\documentclass[numbers=noenddot,12pt,a4paper,notitlepage,twoside,BCOR15mm]{scrartcl}

\usepackage[T1]{fontenc}
\usepackage[utf8]{inputenc}

\usepackage[infoshow]{tabularx}
\usepackage[all]{xy}

\usepackage{amsmath,mathtools}
\usepackage{amssymb}
\usepackage{units}
\usepackage{upgreek}
\usepackage{esint}
\usepackage{graphicx}
\usepackage{ziffer}

\usepackage{float}
\usepackage{lscape}

\usepackage[labelfont=bf]{caption}
\usepackage{wrapfig}
\usepackage{subcaption}

\usepackage[backref=page]{hyperref}

\usepackage{csquotes}
\usepackage[infoshow]{tabularx}
\usepackage{fancyhdr}

\usepackage{sectsty}
\usepackage{times}

\usepackage{lmodern} %TODO Schriftart
\usepackage[greek,ngerman]{babel} %TODO Sprache einstellen

\renewcommand{\headrulewidth}{0.1pt}
\renewcommand{\footrulewidth}{0.1pt}
\newcommand{\name}{\text{Philipp Hacker}} %TODO Name des Protokollanten eintragen

\setlength{\parindent}{0pt}

\newcommand{\degree}{^\circ}
\newcommand{\diff}{\textnormal{d}}
\newcommand{\tenpo}[1]{ 10^{#1}}
\newcommand{\greek}[1]{\greektext#1\latintext}
\newcommand{\ix}[1]{_\text{#1}}
\newcommand{\imag}{\mathbf{i}}
\newcommand{\tilt}[1]{\textit{#1}}
\newcommand{\grad}[1]{\textit{grad}\left(#1\right)}
\newcommand{\divergenz}[1]{\textit{div}\left(#1\right)}
\newcommand{\euler}{\mathnormal{e}}
\newcommand{\fett}[1]{\textbf{#1}}
\newcommand{\ket}[1]{|#1\rangle}
\newcommand{\bra}[1]{\langle#1|}

\title{\fett{\underline{Protokoll: FTIR-Spektroskopie}}} %TODO Name des Versuchs eintragen
\author{Alexander Jankowski, Philipp Hacker}
\date{\today}
\pagestyle{fancy}
\fancyhead[C]{\thepage}
\fancyhead[R]{\name}
\fancyfoot[C]{\thepage}
\fancyhead[L]{Abschnitt \thesection}

\begin{document}

	\renewcommand*{\equationautorefname}{Gl.}
	\renewcommand*{\figureautorefname}{Abb.}
	\renewcommand*{\tableautorefname}{Tab.}
	\renewcommand*{\sectionautorefname}{Abschn.}
	\renewcommand*{\subsectionautorefname}{Abschn.}
	\renewcommand*{\subsubsectionautorefname}{Abschn.}
	\renewcommand*{\figurename}{Abb. }
	\renewcommand*{\tablename}{Tab.}

	\renewcommand*{\figurename}{Abbildung }
	\renewcommand*{\tablename}{Tabelle}

	
	\onecolumn
	\maketitle

	\begin{center}
		Betreuer: U. Martens\\ %TODO Name des Betreuers eintragen
		Versuchsdatum: 28.01.2016 \\ %TODO Datum des Versuchs eintragen
		\begin{table}[h]
			\centering
			Note: %TODO Gute Note erhalten :)
			\begin{tabularx}{1.5cm}{|X|}
				\hline \\ \\
				\hline
			\end{tabularx}
		\end{table}
	\end{center}


	\vspace*{\fill}
	\tableofcontents
	\vfill
	\clearpage
	
	\twocolumn

	\section{Motivation}

		Eine FTIR-Spektroskopie einer Probe kann dazu genutzt werden, quantitativ als auch qualitativ Informationen \"uber die Zusammensetzung dieser zu erhalten.
	
	\section{Physikalische Grundlagen}

		Bei einer FTIR-Untersuchung handelt es sich um die spektroskopische Aufl\"osung von funktionellen Gruppen einer Fl\"ussigkeit/eines Gases mit Hilfe der \tilt{Fourier-Transformations-Infrarotspektrometrie}. Mit einem pr\"azisen Interferometer wird dabei ein Interferogramm - der Verlauf der Interferenzerscheinungen auf dem Schirm des Interferometers, welche durch die \"Uberlagerung von zwei Einzelstrahlen einer IR-Quelle entstehen - aufgenommen, welches dann \"uber eine \tilt{Fourier-Transformation} aus dem Orts- in den Frequenzraum abgebildet wird.\\
		Die bei der FTIR-Spektroskopie benutzte Infrarotstrahlung im Wellenl\"angenbereich zwischen $\unit[[2500-15400]]{nm}$ bzw. den Wellenzahlen $\unit[[4000-650]]{cm^{-1}}$ regt in der, im Strahlengang des Interferometers befindlichen Probe, Molek\"ulschwingungen an. Verschiedene Arten von Molek\"ulen bzw. funktionellen Gruppen dieser, welche wiederum anders in der Probe gebunden sind, haben unterschiedliche Eigenfrequenzen der Schwingungen. Durch diese Anrgegung wird Energie aus der elektromagnischen Welle von der Probe absorbiert, was ein, f\"ur das zu untersuchende Objekt spezifisches Absorptionsspektrum liefert. Der Teil der Welle, welcher nicht absorbiert, sondern einfach transmittiert wird, ergibt wiederum ein einzigartiges Transmissionsspektrum. Diese \tilt{Fingerabdruck-Methode} bedarf einer umfangreichen Datenbank von Korrelationstabellen - Literaturspektren zu bekannten Materialien/Proben - um das eigene Spektrum einordnen und schlie{\ss}lich eine Bestimmung der Molek\"ul(-gruppen) vornehmen zu k\"onnen.
		
	\subsection{Molek\"ulschwingungen}
	
		Das quantenmechanische Modell bedient sich der parabolischen N\"aherung des harmonischen Potentialminimums der Molek\"ulbindungen. Man geht dabei au{\ss}erdem davon aus, dass die Relativbewegung von Atomkernen und Elektronen von den wesentlich schnelleren und leichteren Fermionen bestimmt wird - die tilt{Born-Oppenheimer-N\"aherung}. Des weiteren erh\"alt man aus solchen Schwingungen nur Infrarotstrahlung, wenn ein vorliegendes Dipolmoment sich zeitlich \"andert. Die station\"are Schr\"odingergleichung f\"ur ein Molek\"ul\ im Zustand $\Psi(\vec{x})$ mit den Atomen der reduzierten Massen $\mu$	lautet folglich \cite{FTIRInfra}
		
			\begin{align}
				\Aboxed{
				-\frac{\hbar^{2}}{2\mu}\frac{\diff^{2}\Psi(\vec{x})}{\diff x^{2}}+V(\vec{x})\Psi(\vec{x})=E\Psi(\vec{x})
				}
				\label{eq:schroed}
			\end{align}
	
		F\"ur das Potential $V(\vec{x})=k/2\cdot (\vec{x}-\vec{x}\ix{0})^{2}$ um die Gleichgewichtslage $\vec{x}\ix{0}$ liefert \autoref{eq:schroed} das Ergebnis f\"ur die Energiesniveaus $E_{\nu}$ nach den Schwindungsquantenzahlen $\nu$:
	
			\begin{align}
				\Aboxed{E_{\nu}=(\nu+\frac{1}{2})\cdot\hbar\sqrt{\frac{k}{\mu}}\,\, .
				}
			\end{align}
	
		F\"ur die Vereinfachung, dass die Molek\"ule n\"aherungsweise harmonische Oszillatoren sind (s.o.), so gilt die Auswahlregel f\"ur \"Uberg\"ange in dem erzeugten Schwingungspektrum $\Delta\nu=\pm1$. Die Absorption von einem Photon der Energie $\hbar\omega=\hbar\sqrt{k/m}$ entspricht demnach dem \"Ubergang von $\nu\rightarrow\nu+1$.\\
		Starke chemische Bindungen von Atomen kleiner Massen ben\"otigen gro{\ss}e Schwingungsquanten, schw\"achere Bindungen schwerer Atome kleinere.\\
		Unterschieden werden muss im IR-Spektrum noch zus\"atzlich die Art der Molek\"ulschwingung: Normalschwingungen erzeugen nicht immer Infrarotstrahlung, sondern nur dann, wenn sich innerhalb des System des Molek\"uls L\"angen oder Winkel \"andern. Demnach geh\"oren bspw. Translationen und Rotationen nicht zu den IR-aktiven Schwingungen.
	
	\subsection{Interferometrie und Fourier-Transformation}
	
		Das Interferometer innerhalb des Spektrometers besteht aus einem halbdurchl\"assigen Strahlteiler, einem festen sowie einem beweglichen Spiegel. Die normale Infrarotstrahlung eines - im Idealfall - erhitzten \tilt{schwarzen K\"orpers} \cite{FTIRSpek} mit Wellenzahlen $\unit[[400-7800]]{cm^{-1}}$ (Infrarot $\leftrightarrow$ W\"armestrahlung) wird an dem Strahlteiler aus Kaliumbromid - Transmission bei $\unit[[400-7800]]{cm^{-1}}$ - in zwei Teilstrahlen aufgeteilt. Wie in einem \tilt{Michelson-Interferometer} werden diese dann einerseits auf den feststehende, andererseits auf den beweglichen Spiegel gelenkt. Auf dem selben Strahlteiler kommt es danach zu verschiedenen Interferenzerscheinungen zwischen den reflektierten Strahlen, je nachdem, welche optische Wegdifferenz und Frequenz vorliegt.\\
		Der wieder zusammengef\"uhrte Strahl wird durch eine Blende, den \tilt{J-Stopp}, auf die Probe geleitet. Dort werden die Molek\"ulschwingungen von der Infrarotstrahlung angeregt, woraus die charakteristische Absorption bzw. Transmission in Abh\"angigkeit der Frequenz folgt. Ein \tilt{DTGS}-Detektor - kristallines \tilt{Deuteriertes Triglycinsulfat} hat die g\"unstige pyroelektrische Eigenschaft, das Ladungstrennung bei Temperatur\"anderungen/Verformungen aufgrund polarer Einheitszellen eintritt - nimmt das erhaltene Signal in Abh\"angigkeit der Stellung des beweglichen Spiegels auf. Dieses nicht-korrigierte Interferogramm als Funktion $I\ix{IF}(x)$ von der optischen Wegdifferenz $x$ muss noch um einen konstanten Teil - Beachtung des, an die Quelle verloren gegangenen reflektierten Teils - berichtigt werden. Schlie{\ss}lich erh\"alt man aus einer  \tilt{Cosinus-Fourier-Transformation} aus dem Orts- in den Frequenzraum das Spektrum $I\ix{F}(f)$,
		
			\begin{align}
				\Aboxed{
				I\ix{F}(f)=\int_{-\infty}^{+\infty}I\ix{IF}(x)\cdot\cos(2\pi\nu^{\prime}x)\diff \nu^{\prime}\,\, .
				}
				\label{eq:fourier}
			\end{align}
		
		Im allgemeinen gibt man das Spektrum in Abh\"angigkeit von der Wellenzahl $k=2\pi/\lambda$ an. Demnach findet man die Absorptionsbanden von \tilt{Streckschwingungen} der Molek\"ule des Typs \tilt{X-H}, wobei \tilt{X} ein beliebiges Atom und \tilt{H} Wasserstoff ist, im Bereich von $\unit[[2500-4000]]{cm^{-1}}$. Die st\"arkeren 3-fach Bindungen findet man hingegen im Bereich $\unit[[2000-2500]]{cm^{-1}}$ auf Grund ihrer h\"oheren Kopplungskonstante des Bindungspotentials. Au{\ss}erdem k\"onnen zu diesem Spektralbereich auch X-H-Streckschwingungen mit schweren X-Atomen beitragen. Der Gr\"o{\ss}te Anteil im Bereich zwischen $\unit[[1500-2000]]{cm^{-1}}$ ist den Streckschwingungen der Doppelbindungen mit Kohlenstoffatomen zu zuordnen \cite{FTIRAns}.\\
		Im sogenannten ``Fingerprint''-Bereich zwischen $\unit[[400-1500]]{cm^{-1}}$ manifestieren sich vor allem feinste Unterschiede zwischen \"ahnlichen Molek\"ulen, welche durch chemische unterschiedliche Anbindungen, die Umgebung oder Defekte hervorgerunfen werden. Solche sind s\"amtliche elektronischen Effekte sowie Beuge- und Ger\"ustschwingungen.\\
		Schlie{\ss}lich gilt es zu beachten, dass nat\"urlich auch die Wechselwirkung induzierter/permanenter Dipole in der Probe mit der Infrarotstrahlung eine Ver\"anderung des reinen Schwingungsspektrums hervor ruft.
		
	\subsection{Grenzen und Aufl\"osung der FTIR-Spektroskopie}
	
		Im allgemeinen entspricht ein Interferogramm einer Cosinus-Funktion um einen Ursprungswert $x\ix{0}$ im Ortsraum. Dies gilt f\"ur eine Wellenl\"ange und eine beliebige Anfangsposition des beweglichen Spiegels. Ein Beispiel f\"ur ein reales Interferogramm zeigt \autoref{img:inter}, welches aus dem vorliegenden Experiment stammt. Die Spiegelposition wurde dabei um einen Ausgangswert 0 herum verschoben, welche dem gleichen optischen Weg zum Strahlteiler wie beim festen Spiegel entspricht.
		
		\begin{figure}[h]
			\centering
			\includegraphics[width=0.4\textwidth]{Gruppe2A/inter.pdf}
			\caption{Interferogramm aus dem FTIR-Interferometer ohne Transformation in den Frequenzraum und Probe.}
			\label{img:inter}
		\end{figure}
	
		Die sog. \tilt{Trunkation} - das Abschneiden des Interferogramms bei der maximalen Verschiebung des Spiegels - liefert nach einer Fourier-Transformation eine Spektralfunktion der Form $\propto\sin(x)/x$, dessen Seitenbanden ung\"unstig f\"ur weitere Untersuchungen der Probe sind. \"Uber eine \tilt{Apodisation}, dh.die Faltung mit einer weiteren Funktion bspw. mit Rampen-Form, verbreitert man zwar den Peak der Spektralfunktion, unterdr\"uckt dabei aber zus\"atzliche Extrema nebem dem Hauptmaximum. Das Ergebnis, ein Graph $\propto\sin^{2}(x)/x^{2}$ entspricht \autoref{img:inter}.\\
		Im Hinblick auf das Aufl\"osungsverm\"ogen $\Delta\nu$ eines FTIR-Spektrometers findet man, dass dieses ma{\ss}geblich von der maximalen Auslenkung $L$ des beweglichen Spiegels und dem Winkel, unter welchem der Detektor die Quelle betrachtet, beeinflusst wird. Eine optimale Aufl\"osung stellt sich mit $\Delta\nu\approx1,3/2L$ ein, wobei allgemein
		
		\begin{align}
			\Aboxed{
			\frac{\nu}{\Delta\nu}=2L\nu
			} \nonumber
		\end{align} 
		
		gilt. Hohe Aufl\"osungen von $\unit[1]{cm^{-1}}$, f\"ur bspw. Betrachtungen von Verunreinigungen etc., gehen mit einem gr\"o{\ss}eren Rausch-zu-Signal-Verh\"altnis (\tilt{signal-to-noise-ration} SNR) einher. Entsprechend muss sich im Vorfeld einer Untersuchung f\"ur eine sinnvolle Aufl\"osung bei einer gegebenen Probe entschieden werden.\\
		Ein Vorteil der FTIRS ist, dass fast die vollst\"andige Intensit\"at der IR-Quelle zur Probe geleitet werden kann (nur J-Stopp). Im Gegensatz zu Gitterspektrographen o.\"a. ist die Energie am Detektor um ein vielfaches h\"oher. Der \tilt{Jacquinot-Vorteil} ergibt sich zu
		
		\begin{align}
			\Aboxed{
			\frac{(SNR)\ix{FT}}{(SNR)\ix{G}}\approx200\,\,.
			} \nonumber
		\end{align} 
		
		Daraus ergibt sich zudem der \tilt{Felgett-Vorteil}, welcher auf der gleichzeitigen Messung von vielen Frequenzen $\nu$ beruht. Er gibt den Vorteil des SNR eines, in einer kurzen Zeit gemessenen Signals an und ist proportional zur Wurzel aus der Menge der aufgel\"osten Elemente $M$ \cite{FTIRAns}.
		
		\begin{align}
			\Aboxed{
			\frac{(SNR)\ix{FT}}{(SNR)\ix{G}}=\sqrt{M}
			} \nonumber
		\end{align}
		
		Die Wellenzahlskala kann \"uber die Kalibrierung mit einem \tilt{He-Ne-Referenzlaser} vorgenommen werden. Damit ist die permanente Kontrolle der optischen Wegdifferenz mit den Interferenzerscheinungen des Lasers sichergestellt. Diese Abtastung des Interferogramms mit der monochromatischen Lichtquelle ergibt den \tilt{Connes-Vorteil}.
		
		\subsection{Quantitative Analyse}
		
		Eine Untersuchung der Absorption im Hinblick auf die statistische Zusammensetzung von Mehrkomponentenproben ist \"uber einen Ansatz nach \tilt{Lamber-Beer} m\"oglich. Bei einer Konzentration des Stoffes in der Probe $c$, deren optischer Wegl\"ange $d$ und dem Absorptionskoeffizienten $\lambda(\nu)$ bei der Wellenl\"ange $\nu$, wird die Absorption $A$ zu
		
		\begin{align}
		\Aboxed{
			\log\left(\frac{I\ix{0}}{I}\right)=A=\lambda(\nu)cd\,\, .
		}
		\label{eq:lamber}
		\end{align}
		
		
		Dabei ist $I\ix{0}$ die Intensit\"at vor der Probe und $I$ auf dem Detektor. Der Koeffizient $\lambda(\nu)$ ist eine Molek\"u-spezifische Gr\"o{\ss}e. Insgesamt ist die Absorption der Probe additiv bez\"uglich der enthaltenen Molek\"ule in der gemischten Probe. Sind die Konzentrationen nicht zu hoch - sonst treten Ver\"anderungen der Absorptionskoeffizienten auf Grund von st\"arkeren Wechselwirkungen auf - kann bei einer monochromatischen Strahlung so in guter N\"aherung die anteilige Zusammensetzung bestimmt werden.\\
		H\"ohere Gasdr\"ucke sorgen f\"ur eine Verbreiterung der IR-Absorptionsbanden, weil verst\"arkt Wechselwirkungen die Anregung der Molek\"ule dominieren. Dies gilt f\"ur alle charakteristischen Banden.\\
		Steigt die Temperatur der Probe, so ver\"andert sich folglich die thermische Gleichverteilung der Molek\"ule in den Schwingungsniveaus. Die spezifische Absorption der Probe ver\"andert sich und zeigt u.U. ganz neue Peaks. 
		
	\subsection{ATR-Infrarotspektroskopie}
		
		Die \tilt{abgeschw\"achte Totalreflexion} (engl.: \tilt{attenuated total reflection} ATR) wird genutzt, um sehr viskose Fl\"ussigkeiten - also bspw. Gele - Pulver oder Folien mit der Infrarotspektroskopie zu untersuchen.\\
		\"Uber einen Lichtwellenleiter - in unserem Fall einen Diamanten, in welchem Mehrfachreflexion m\"oglich ist - mit sehr hohem Brechnungsindex, wird die IR-Strahlung, idealerweise aus einer Schwarzk\"orper-Quelle, auf die Probe geleitet. Dort wird sie am optisch D\"unneren Medium der Probe zur Totalreflexion gebracht. Klassisch betrachtet wird das eingestrahlte Licht bei der Totalreflexion an der Grenzfl\"ache zum Substrat vollst\"andig reflektiert. Tats\"achlich bilden sich jedoch \tilt{evaneszente Wellen} in der Probe aus, wenn sich die elektromagnetische Welle unter dem eingestrahlten Winkel nicht ausbreiten kann. Deren Amplitude f\"allt stetig, jedoch schnell mit der Eindringtiefe, als Folge von quantenmechanischen Anschlussbedingungen ab \cite{FTIRInfra}.\\
		Auf diesem Weg wird der elektromagnetischen Welle der IR-Strahlung Energie entzogen. Ein Ma{\ss} f\"ur die Absorption der Welle bei der ATR ist die Eindringtiefe $d\ix{P}$, bei der die Amplitude auf das  $1/\euler$-fache abgefallen ist. Bei der Wellenl\"ange $\lambda$, den Brechnungindizes $n\ix{D}$ des Leiters und $n\ix{P}$ der Probes, sowie unter dem Einstrahlwinkel $\varphi$ ergibt sich diese zu
		
		\begin{align}
		\Aboxed{
			d\ix{P}=\frac{\lambda}{2\pi n\ix{D}\sqrt{\sin^{2}(\varphi)-(n\ix{P}/n\ix{D})^{2}}}\,\, .
		} \label{eq:atr}
		\end{align}
		
		Wichtig bei der ATR ist die hohe Dichte der Probe vor dem Lichtwellenleiter, um ein gutes Signal zu erhalten. Man erh\"alt als Messsignal die Absorption in Abh\"angigkeit der Wellenl\"ange, nimmt man parallel das von der Probe reflektierte Licht wiederum zuerst wieder als Interferogramm auf (siehe \autoref{eq:fourier}).

			\begin{figure*}
			\centering
				\includegraphics[width=0.85\textwidth]{Gruppe2A/methan.pdf}
				\caption{Absorptionsspektrum der FTIRS einer Methan-Gaszelle. Subtrahierte Basislinie eingetragen.}
				\label{img:methan}
			\end{figure*}

	\section{Durchführung}

		Im Rahmen der FTIR-Spektroskopie dieses Versuches werden verschiedene Gase, Fl\"ussigkeiten, Pulver und Folien unter zur Hilfenahme der ATR-Einheit untersucht. Dabei muss zudem R\"ucksicht auf die Transmissions- und Absorptionseigenschaften der verwendeten Probenzellen, sowie deren Vertr\"aglichkeit mit den jeweiligen Inhalten genommen werden.\\
		In der Gasphase gilt es Methan und Stickstoff zu analysieren und mit Literaturbilbiotheken f\"ur spezifische, experimentell sondierte Absorptionsbanden zu vergleichen. In einer Fl\"ussigkeitszelle mit Fenstern aus \fett{CaF}$_{\fett{2}}$ werden dann anschlie{\ss}end 3 \"Ole - aus Oliven, Sonnenblumen und Erdn\"ussen - im FTIR-Spektrometer untersucht. Zuletzt wird die ATR-Einheit in das Interferometer eingesetzt, damit mit dessen Hilfe weitere 3 unbekannte Pulver spektroskopisch untersucht werden k\"onnen.\\
		Im Vorfeld zu jeder Art von Messung ist es notwending, ein Untergrundspektrum - mit anderen Worten ein Interferogramm, wie in \autoref{img:inter} ohne Probe - zu Korrektur- und Kalibrationszwecken aufzunehmen.\\
		Bei der Arbeit mit den Fl\"ussigkeitszellen ist zu beachten, das unterschiedliche Abstandshalter zwischen Zelle und Glasscheibe (\tilt{spacer}) eingesetzt werden. Es liegen Aluminiumfolie und Teflon-Spacer vor.\\
		Des weiteren gilt Vorsicht beim Einf\"ullen der Proben in die K\"uvetten: Luftblasen oder eingeschlossene \"Uberreste anderer Untersuchungen k\"onnen die Absorption stark beeintr\"achtigen oder verf\"alschen, siehe \cite{FTIRAns},\cite{EMAUGreifswaldFTIR}.

	\section{Auswertung}

		\subsection{Untersuchung der Gase}

		\subsubsection*{Stickstoff}
		Zuerst wurde in einer Gaszelle mit Kaliumbromid-Fenstern (\fett{KBr}) Stickstoff untersucht. Zwischen $\unit[450]{cm^{-1}}$ und $\unit[4000]{cm^{-1}}$ ist in $\unit[1]{cm^{-1}}$ eine Serie von 10 Scans aufgenommen worden. Um $\unit[670]{cm^{-1}}$ zeigte sich eine Absorptionsbande, die, der Literatur zur Folge, Polysterene-Beh\"altern zugeordnet werden kann, in welchen der Stickstoff transportiert bzw. aufbewahrt und gemischt wird. Ein weiterer Einfluss gelangt \"uber die verwendeten Fenstermaterialien in das Spektrum: bei etwa $\unit[1000]{cm^{-1}}$  liegt eine Bande, welche der Wechselwirkung anorganischer Proben mit \fett{KBr} zugeordnet werden kann.\\
		Leider ist eine weitere Analyse der Messdaten und eine Darstellung nicht m\"oglich, da betreffende Datein aus der propiet\"aren Software nicht entnommen wurde.
		
		\subsubsection*{Methan}
		Hierbei wurde gleichsam, wie f\"ur den Stickstoff verfahren. Das Spektrum und dessen subtrahierte Basislinie werden in \autoref{img:methan} gezeigt.\\
		Im Absorptionsspektrum sind zum einen gro{\ss}e Peaks und mehrere Substrukturen um diese bei $\unit[1350]{cm^{-1}}$ und $\unit[3000]{cm^{-1}}$ zu sehen. Der erste Peak kann Bindungen der Art -[\fett{CH}$_{\fett{2}}$]$_{N}$- zugeordnet werden. Des weiteren k\"onnen  u.a. sulfatische S\"auren mit Gruppen der Form -[\fett{SO}$_{\fett{2}}$]- zu dieser Absorption beitragen. Offensichtlich z\"ahlt dieser Einfluss damit zu denen der Verunreinigungen. Au{\ss}erdem werden in der Literatur auch Gruppen von -[\fett{NO}$_{\fett{2}}$]- f\"ur diesen Bereich verzeichnet. Ein kleinerer Nebenpeak um ca. $\unit[1550]{cm^{-1}}$ l\"asst sich besonders gut mit den $\Delta\nu=0,\pm1$ \"Uberg\"angen der P-,Q-, und R-Zweige (siehe \autoref{eq:schroed}) der -[\fett{C=C}]- Gruppe identifizieren.\\
		Das zweite, gro{\ss}e Absorptionsmaximum zeigt \"ahnliche Merkmale - eine Dreiteilung in die Zweige der entsprechenden \"Uberg\"ange zwischen den Niveaus - und kann den Schwingungen von verschiedenen -[\fett{C-H}]- Gruppen zugesprochen werden. Dem breiten Verlauf dieser Bande in Richtung $\unit[2500]{cm^{-1}}$ sind Schwingungen von Verunreinigungen mit -[\fett{O-H}]- Gruppen geschuldet.

		\subsection{Fl\"ussigkeitsuntersuchungen}

		\subsubsection{Schichtdickenbestimmung}

			\begin{figure*}
			\centering
				\includegraphics[width=0.85\textwidth]{Gruppe2A/erdnuss.pdf}
				\caption{Transmissionsspektrum einer Fl\"ussigkeitsprobe mit Erdnuss-\"Ol. Basislinie und zweites Spektrum mit Abstandshaltern (\tilt{spacer}) eingetragen. Aus der Datenbibliothek (propiet\"ar) erhielt man eine 97.8\%-ige \"Ubereinstimmung in den Absorptionsbanden mit Methyl-Linoleat (\tilt{Linols\"auremethylesther}).} 
				\label{img:erdnuss}
			\end{figure*}

		Zur Bestimmung und sinnvoller Korrektur von Spektren mit Fl\"ussigkeitsk\"uvetten kann eine Schichtdickenermittlung der verwendeteten Bauteile vorgenommen werden. Das Interferenzmuster einer leeren K\"uvette kann demnach genutzt werden, welche Dicke die absorbierenden Komponenten dieser haben.\\
		Im vorliegenden Fall handelt es sich dabei um die Abstandshalter (\tilt{spacer}) zwischen den Halterungen und Fenstern des Fl\"ussigkeitsbeh\"alters. \"Uber die Formel
		
		\begin{align}
		\Aboxed{
			d/\unit{\upmu m}=\frac{N\cdot 10000}{2n(\nu\ix{1}-\nu\ix{2})}
			}
		\end{align}
		
		kann die Schichtdicke der spacer berechnet werden.

		\begin{figure}[h]
			\centering
			\includegraphics[width=0.4\textwidth]{Gruppe2A/spacer.pdf}
			\caption{Transmissions der \tilt{Spacer}.}
			\label{img:space}
		\end{figure}
		
		Dabei ist N die Zahn der vollst\"andigen Wellenz\"uge zwischen den Wellenzahlen $\nu\ix{i}$, wenn innerhalb der K\"uvette ein Medium mit dem Brechnungsindex $n$ vorliegt. F\"ur eine d\"unne Aluminium-Folie wurde die Dicke zu $d\ix{Al}=\unit[79.67]{\upmu m}$, und f\"ur Teflon zu $d\ix{Tf}=\unit[28.92]{\upmu m}$ ermittelt. Die Leer-Spektren zeigt \autoref{img:space}.

		\subsubsection*{Erdnuss\"ol}

		Die \autoref{img:erdnuss} zeigt das Ergebnis der Spektroskopie von Erdnuss-\"Ol. Das korriegierte Spektrum erh\"alt man, wenn die spacer f\"ur die Untersuchung entfernt werden und wenn davon die Basislinie subtrahiert wird. In diesem Fall werden die hohen Banden nicht mehr von der Apparatur in ihrer Aufl\"osung abgeschnitten.\\
		In einem Vergleich zur Literatur findet man in automatisierten Bibliotheken eine \"Ubereinstimmung von 97.8\% mit einem Esther der Linol-S\"aure. Des weiteren gibt es Korrelationen \"uber 95\% mit Methyl-Meristat, welches sich jedoch gleichwertig im Spektrum wie der erste Fall ausdr\"uckt.\\
		\tilt{Linols\"auremethylesther} hat die Summenformel \fett{C}$_{\fett{19}}$\fett{H}$_{\fett{34}}$\fett{O}$_{\fett{2}}$ und besitzt zwei \fett{CH}$_{\fett{3}}$ Gruppen. Die entsprechende Absorptionsbande findet sich bei $\unit[3150]{cm^{-1}}$ im Spektrum wieder. Au{\ss}erdem tragen auch im Bereich $\unit[2800-3000]{cm^{-1}}$ die vielen \fett{CH}$_{\fett{2}}$ bei. Diese -[\fett{CH}$_{\fett{2}}$]$_{N}$- sind auch wiederum Ursache f\"ur die Strukturen bis $\unit[1500]{cm^{-1}}$. Der scharfe Peak bei ca. $\unit[1750]{cm^{-1}}$ ist Ergebnis der Schwingungen der einfachen -$[\fett{CO}]$- und der einen, doppelten -$[\fett{C=O}]$- Bindung.
		
		\subsubsection*{Sonnenblumen-\"Ol}
		
			\begin{figure*}
			\centering
				\includegraphics[width=0.85\textwidth]{Gruppe2A/sonnen.pdf}
				\caption{Transmissionsspektrum einer Fl\"ussigkeitsprobe mit Sonnenblumen-\"Ol. Das zweite Spektrum mit Abstandshaltern (\tilt{spacer}) ist gezeigt. Aus der Datenbibliothek (propiet\"ar) erhielt man eine 95.18\%-ige \"Ubereinstimmung in den Absorptionsbanden mit Methyl-Myristat.} 
				\label{img:sonnen}
			\end{figure*}
		
		Die \autoref{img:sonne} enth\"alt die rohen und korrigierten Spektren von Sonnenblumen-\"Ol sowie das Literatur-Ergebnis zu \tilt{Methyl-Myristat}, welches damit eine 95.18\%-ige \"Ubereinstimmung haben. Dieses hat die Summenformel \fett{C}$_{\fett{15}}$\fett{H}$_{\fett{30}}$\fett{O}$_{\fett{2}}$. Im Vergleich zum Spektrum aus \autoref{img:erdnuss} sieht man schnell, dass alle Annahmen zu diesen Ergebnissen auch hier greifen. Besonders anschaulich wird diese \"Ubereinkunft, wenn man die Strukturformel (siehe \cite{FTIRStruk} o.\"a.) miteinander vergleicht. Beide enthalten 2 -[\fett{CH}$_{\fett{3}}$]- sowie mehrer Kohlenstoff-Doppelbindungen. Zudem findet man bei beiden Verbindungen eine Gruppe der Form -[\fett{O-C=O}]-.
		
		\subsubsection*{Oliven-\"Ol}

		Die \autoref{img:oliven} skizziert nochmals das Spektrum f\"ur Oliven\"ol und die h\"ochste Korrelation, \fett{C}$_{\fett{18}}$\fett{H}$_{\fett{36}}$\fett{O}$_{\fett{2}}$ Ethyl-Palmitat. Es folgen die Gleichen Argumente wie f\"ur \autoref{img:sonnen} und \autoref{img:erdnuss}.

			\begin{figure}[h]
			\centering
				\includegraphics[width=0.4\textwidth]{Gruppe2A/oliven.pdf}
				\caption{Absorptionsspektrum einer Oliven-\"Ol Probe. Vergleich mit dem am besten \"ubereinstimmenden Bibliothekseintrag, Ethyl-Myristat.}
				\label{img:oliven}
			\end{figure}

		Die Ausnahme an diesem \"Ol ist, dass keine Methyl-Gruppen enthalten sind. Nur eine -[\fett{O-C=O}]- Komponente sorgt f\"ur den scharfen Peak um $\unit[1750]{cm^{-1}}$. Die weiteren 12 -[\fett{CH}$_{\fett{2}}$]- Gruppen erzeugen die Banden zwischen $\unit[2700-3100]{cm^{-1}}$.
		
		\subsection{ATR-Untersuchungen}
		
		\subsubsection*{Folien}

			\begin{figure*}
			\centering
				\includegraphics[width=0.85\textwidth]{Gruppe2A/folien.pdf}
				\caption{} 
				\label{img:folien}
			\end{figure*}

		Bei den Untersuchungen mit der ATR-Einheit wurde mit einer unbekannten Folie begonnen, siehe \autoref{img:folien}. Nach Analyse mit Hilfe der Daten-Bibliothek fand man leicht heraus - Korrelation zu 98.98\%, mit dem blo{\ss}en Auge im Spektrum kaum zu unterscheiden - dass diese vollst\"andig aus Polyethylen bestand. Die Banden - bei $\unit[2800]{cm^{-1}}$, $\unit[1450]{cm^{-1}}$ sowie $\unit[800]{cm^{-1}}$ - sind deswegen den $N$-fachen Ethylen-Ketten -[\fett{C}$_{\fett{2}}$\fett{H}$_{\fett{4}}$]$_{N}$- zuzuordnen. Eine Substruktur um $\unit[2800]{cm^{-1}}$ deutet auf eine Aufteilung in die Banden zu $\Delta\nu\pm1$ hin.\\
		Im zweiten Schritt untersuchte man bewusste \tilt{PVC}, also ebenso eine Folie aus Anordnungen der Form -[\fett{C}$_{\fett{2}}$\fett{H}$_{\fett{3}}$\fett{Cl}]$_{N}$-. Unerwarteter Weise sieht die charakterische Absorption dieser Molek\"ulgruppe fast vollkommen anders aus. 

	\section{Anhang}

		\bibliography{all.bib}
		\bibliographystyle{unsrt}

\end{document}