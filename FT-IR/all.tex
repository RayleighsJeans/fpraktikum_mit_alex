@article{Wiki:PET,
	title={{Wikipedia-Artikel zum Thema: Technik der Positronen-Emissions-Tomographie, Abschnitt 3: Wunsch und Wirklichkeit: Was will man messen und was wird gemessen?}},
	journal={Wikipedia: Die freie Enzyklopädie},
}

@article{Wiki:AFM,
	title={{Wikipedia-Artikel zum Thema: Rasterkraftmikroskopie}},
	journal={Wikipedia: Die freie Enzyklopädie},
}

@article{EMAUGreifswaldPKO,
	author={Praktikum für Fortgeschrittene, Praktikumsanleitung},
	title={Versuch 02: Positronen und Koinzidenzmessungen},
	journal={Ernst-Moritz-Arndt-Universität Greifswald, Institut für Physik},
}

@article{EMAUGreifswaldMIE,
	author={Praktikum für Fortgeschrittene, Praktikumsanleitung},
	title={Versuch 03: Mie-Streuung},
	journal={Ernst-Moritz-Arndt-Universität Greifswald, Institut für Physik},
}

@article{TUDarmstadtPET,
	author={Praktikum für Fortgeschrittene, Praktikumsanleitung},
	title={Versuch 2.8-B:Grundlagen der Positronen-Emissions-Tomographie(PET)},
	journal={TU Darmstadt, Institut für Kernphysik},
}

@article{USiegenMIE,
	author={ Prof. Pietsch, M. Alhumaidi},
	title={ Röntgenreflektometrie/X-Ray reflection},
	journal={Universität Siegen, Arbeitsgruppe für Festkörperphysik, Masterpraktikum - Wintersemester 2015/2016},
}

@article{EMAUGreifswaldReflektron,
	author={Praktikum für Fortgeschrittene, Praktikumsanleitung},
	title={Versuch 04: Reflektron und tof-Massenspektrometrie},
	journal={Ernst-Moritz-Arndt-Universität Greifswald, Institut für Physik},
}

@article{UMainzReflektron,
	title={Die Paulfalle - Eine kurze Einführung},
	journal={Universität Mainz, Institut für Physik},
	note={Zu finden unter: \url{http://www.physik.uni-mainz.de/werth/calcium/ca_ptrap.html}, 15.11.2015},
}

@article{EMAUGreifswaldOHRot,
	author={Praktikum für Fortgeschrittene, Praktikumsanleitung},
	title={Versuch 05: OH-Rotationsspektroskopie},
	journal={Ernst-Moritz-Arndt-Universität Greifswald, Institut für Physik},
}

@article{WikiRotat,
	title={Mirkowellenspektroskopie},
	journal={Wikipedia, die freie Enzyklopädie},
	note={Zu finden unter: \url{https://de.wikipedia.org/wiki/Mikrowellenspektroskopie}, 23.11.2015},
}

@article{EMAUGreifswaldNMR,
	author={Praktikum für Fortgeschrittene, Praktikumsanleitung},
	title={Versuch 05: Kernspinresonanz-Spektroskopie},
	journal={Ernst-Moritz-Arndt-Universität Greifswald, Institut für Physik},
}

@article{TeachSpinNMR,
	author={TeachSpin},
	title={Geräteanleitung eines TeachSpin Kernspinresonanz-Spektroskopie Aufbaus},
	note={Kapitel I.Introduction, Abschnitt B.2 Theory, Rev.1.1 4/09},
}

@article{UAchenNMR,
	author={RWTH Aachen, Institut für Physik},
	title={Versuchsanleitung Physikalisches Praktikum zu NMR},
	note={zu finden unter \url{http://institut2a.physik.rwth-aachen.de/de/teaching/praktikum/Anleitungen/NMRscript.pdf}},
}

@article{SpinEcho,
	author={E. L. Hahn},
	title={Spin Echoes},
	journal={Physical Review, Volume 80 Number 4},
	location={University of Illinois, Urbana, Illinois, Physics Department},
	date={Recieved May 22,1950},
}


@article{Paul-FalleREF,
	author={H. Leuthner, G. Werth},
	title={Buffer-gas-cooled ion clouds in a classical Paul trap: Superimposed stability diagrams and trapping capacity investigations},
	journal={Applied Physics B},
	note={(Impact Factor: 1.86). 10/2013; 114(1-2):421-. DOI: 10.1007/s00340-013-5657-1},
}

@misc{EarnPaul,
	author={S. Earnshaw},
	title={On the nature of the molecular forces which regulate the constitution of the luminiferous ether},
	journal={Trans. Camb. Phil. Soc, 7:97-112, 1842},
}

@misc{Paul,
	author={W. Paul},
	title={Ein Ionenkäfig, Forschungsberichte des Wirtschafts- und Verkehrsministeriums},
	journal={Nordrhein-Westfalen, (451), 1958},
}

@misc{Bruch,
	author={R. March, J. Todd},
	title={Practical Aspects of Mass Spectrometry},
	journal={1. Au, CRC, Press, New York},
	year={1995},
}

@article{Bruch2,
	author={F. Roßbach},
	title={Nichtlineare Dynamik in der Paulschen Ionenfalle},
	journal={1995},
}

@article{EMAUGreifswaldPaul,
	author={Praktikum für Fortgeschrittene, Praktikumsanleitung},
	title={Versuch 06 : Paul-Falle},
	journal={Ernst-Moritz-Arndt-Universität Greifswald, Institut für Physik},
}

@misc{WikiMZInt,
		howpublished={From Wikipedia, the free encyclopedia, Online, siehe URL},
	title={Mach–Zehnder interferometer},
	url={https://en.wikipedia.org/wiki/Mach%E2%80%93Zehnder_interferometer},
	year={am 04.01.2016},
}

@misc{MZwork,
	author={K.P.Zetie, S.F.Adams, R.M.Tocknell},
	year={2000},
	note={Physical Education, SW1 3PB},
	title={How Does a Mach-Zehnder interferometer work?},
}

@book{MZdemt,
	author={Wolfgang Demtröder},
	year={2006},
	publisher={Springer; Berlin Heidelberg, Springer-Verlag},
	title={Experimentalphysik 2, Elektrizität und Optik},
	isbn={978-3-540-33795-9 (Online)},
	pages={231-239},
	note={insbesondere 293, 8.4.8}
}

@book{MZcohe,
	author={W. Lauterborn, T. Kurz},
	year={2002},
	publisher={Springer; Berlin Heidelberg, Springer-Verlag},
	title={Coherent Optics - Fundamentals and Applications},
	pages={38, 39},
	note={Kapitel 4., Coherence}
}

@misc{MZquote,
	author={Abner Schimony},
	year={1996},
	note={In: Neuser, W.; Oettingen, K. N. (Hrsg.): Quantenphilosophie},
	title={Die Realität der Quantenwelt},
	howpublished={Spektrum Akad. Verlag. Heidelberg, S. 70 – 77},
}

@misc{MZaufbau,
	edition={TEP 2.2.08-00},
	publisher={Phywe, TESS-expert},
	title={Quantum Eraser Manual},
	howpublished={PHYWE Systeme GmbH \& Co. KG},
}

@misc{XPSAuger,
	howpublished={Online at 14.01.2016; Wiki.etup.edu: *Auger Electron Spectroscopy (AES)*, \url{https://wiki.utep.edu/pages/viewpage.action?pageId=51217142}},
	publisher={Jose Chavez},
}

@book{XPSRapha,
	author={D. Bruhne, R. Hellborg, H. J. Whitlow and O. Hunderi},
	year={1997},
	publisher={Wiley-VCH Verlag},
	title={Surface Characterization: A User’s Sourcebook},
}

@book{XPSalt,
	author={C. D. Wanger, W. M. Riggs, L. E. Davis, J. F. Moulder and G. E.Muilenberg},
	publisher={Perkin-Elmer Corp., Physical Electronics Division},
	title={Handbook of X-ray Photoelectron Spectroscopyk},
}

@misc{XPSRa,
	howpublished={M. Himmerlich. Photoelectron Emission Microscopy and Photoelectron Spectroscopy of Ge on Si, InN and InP. Diplomarbeit, 1999.},
}

@misc{FTIRSpek,
	howpublished={Online bei : Wikipedia, Die freie Enzyklop\"adie; FTIR-Spektrometer, \url{https://de.wikipedia.org/wiki/FTIR-Spektrometer}},
	date={18.02.2016},
}

@misc{FTIRInfra,
	howpublished={Online bei : Wikipedia, Die freie Enzyklop\"adie; Infrarotspektroskopie, \url{https://de.wikipedia.org/wiki/Infrarotspektroskopie}},
	date={18.02.2016},
}

@misc{ATRInfra,
	howpublished={Online bei : Wikipedia, Die freie Enzyklop\"adie; ATR-Infrarotspektroskopie, \url{https://de.wikipedia.org/wiki/ATR-Infrarotspektroskopie}},
	date={18.02.2016},
}

@misc{FTIRAns,
	publisher={Ansyco; analytische Systeme und Componenten GmbH},
	title={Theorie der FT-IR Spektroskopie},
	howpublished={Ansyco GmbH, Ostring 4, D 76131 Karlsruhe},
}

@article{EMAUGreifswaldFTIR,
	author={Praktikum für Fortgeschrittene, Praktikumsanleitung},
	title={Versuch 11 : FTIR-Spektroskopie},
	journal={Ernst-Moritz-Arndt-Universität Greifswald, Institut für Physik},
}

@misc{FTIRStruk,
	howpublished={NOP-Online, UNI-Bremen, zu finden unter \url{kriemhild.uft.uni-bremen.de/nop/de-substance-176}},
	date={21.02.2016},
}